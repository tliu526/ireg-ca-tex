\documentclass[a4paper, 11pt]{article}
\usepackage{amsmath,amsfonts,amsthm,amssymb,amscd}
\usepackage[top=0.5in, bottom=0.5in, left=0.5in, right=0.5in] {geometry}
%\usepackage{fixltx2e}
\usepackage[numbers]{natbib}
\usepackage{enumitem}

\newcommand{\tab}[1]{\hspace{.2\textwidth}\rlap{#1}}

\begin{document}
\title{Computation in the Wild: Reconsidering Dynamic Systems in Light of Irregularity}
\author{Tony Liu}
\maketitle

\section*{Thesis Outline}

\subsection*{Introduction}
\begin{itemize}
\item Motivating example: \citeauthor{pe04}'s plant cell stomatal coordination
\item Metrics: \citeauthor{la90}'s $\lambda$ and measures of robustness
\end{itemize}

\subsection*{Previous Work}
\begin{itemize}

\item Motivation and Applications
\begin{itemize}
\item Stomatal Patchiness
\item Inspiration for Novel Computing Models
\end{itemize}

\item Criticality and the Emergence of Computation
\begin{itemize}
\item Edge of Chaos
\item Metrics
\end{itemize}

\item Robustness in Face of Irregularity
\begin{itemize}
\item Redundancy in Decentralized Systems
\item Alternative Spatial Representations of CA (Penrose, Voronoi)
\end{itemize}
\item Summary

\end{itemize}

\subsection*{Adapting Criticality Measures to Irregular Grids}
\begin{itemize}

\item $\lambda$ Rule Table Mappings to Irregular Grids
\begin{itemize}
\item Preserving ``directionality'' with both Moore and von Neumman neighborhoods
\item Mapping variable neighborhood sizes to the fixed neighborhood sizes of $\lambda$ rule tables
\end{itemize}

\item Transition Rule Mappings to Irregular Grids
\begin{itemize}
\item Conway's Life: weighting neighborhood influence by Euclidean distance or edge length
\item Mapping Local Majority and 2DGKL transition rules to variable neighborhood sizes
\end{itemize}

\item $\gamma$ and Perpetual Disequilibrium (PD)

\end{itemize}

\subsection*{System Design}
\begin{itemize}

\item Grid Generation
\begin{itemize}
\item Control over density, periodicity, and orientation of generated grids

\item Delaunay Triangulation/Voronoi Diagrams

\item Penrose Tilings (Thin/Thick Rhombs and Kites/Darts)

\item Voronoi Quadrilaterals
\end{itemize}

\item Stencil Mapping

\begin{itemize}
\item Translation between regular and irregular Rule Table Transitions

\item Orientation-Based Stencils

\item Distance-Based Stencils
\end{itemize}

\item Simulation (Optimizations)

\item Reference to Code?

\end{itemize}


\subsection*{Experiment Goals}
\begin{itemize}

\item Voronoi Diagrams
\begin{itemize}
\item GoL: Do we see an increased difficulty for supporting coherent Game of Life structures on Voronoi diagrams?
\item Testing Interconnectivity: Are there differences in CA behavior on Voronoi grids simply due to different average connectivity or because of other properties?
\item Equivalence Classes: Can some sort of equivalence between neighborhood definitions of regular and irregular grids be established?
\end{itemize}

\item Penrose Life
\begin{itemize}
\item Can we replicate the results of \citeauthor{hi05} and \citeauthor{ow10}?
\item Are there discernable differences in CA behavior between periodic and aperiodic tilings?
\end{itemize}

\item Majority Task
\begin{itemize}
\item Do CA rule tables that solve the majority task on regular grids robustly still do so on irregular grids?
\item Do the same effects of temporal and spatial noise \citeauthor{me07} aiding task performance occur witihn the irregular grids?
\end{itemize}

\item Physically Based Systems
\begin{itemize}
\item Artificial Chemistries respect materiality: can we create or evolve CAs that follow ``physical'' rules?
\end{itemize}

\item Algorithms Utilizing Spatial Components
\begin{itemize}
\item Spatially Distributed Genetic Algorithms
\end{itemize}

\end{itemize}

\subsection*{Results}

\begin{itemize}
\item Majority Task
\begin{itemize}
\item Initial results show that local majority task performance on irregular Voronoi diagrams are equivalent to local majority task performance on regular grids
\end{itemize}

\item Penrose Life
\begin{itemize}
\item Replication of Penrose experiments on both Kite/Dart and Thin/Thick Rhombus Penrose tilings agree with the results produced by \citeauthor{ow10}
\item Initial results indicate both a quantitative and qualitative difference between Kite/Dart and Rhomb grids in terms of oscillating structures, ash, and average lifetimes
\end{itemize}

\end{itemize}

\subsection*{Discussion}

\bibliographystyle{plainnat}
\bibliography{../Liu_References}

\end{document}